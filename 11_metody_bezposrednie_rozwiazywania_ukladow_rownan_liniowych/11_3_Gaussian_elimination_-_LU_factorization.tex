\section{Gaussian elimination-LU factorization}
\begin{frame}{Gaussian elimination}
$Ax=b\rightarrow Ux=c, Lc=b$

$\mathrm{U}$ -upper triangular, $\mathrm{L}$ -lower triangular
\scalebox{0.92}{
$\left(\begin{array}{lll}
a_{11} & a_{12} & a_{13}\\
a_{21} & a_{22} & a_{23}\\
a_{31} & a_{32} & a_{33}
\end{array}\right) \left(\begin{array}{l}
x_{1}\\
x_{2}\\
x_{3}
\end{array}\right)=\left(\begin{array}{l}
b_{1}\\
b_{2}\\
b_{3}
\end{array}\right)
\begin{array}{l}
\frac{a_{21}}{a_{11}} 
\\
\\
\\
\end{array}
\begin{array}{l}
\bigg\}-
\\
\\
\end{array}
\left. \begin{array}{l}
\frac{a_{31}}{a_{11}}
\\
\\
\\
\end{array}\right\}
-a_{11}\neq0
$}
$$
\scalebox{0.90}{
$
\left(\begin{array}{lll}
a_{11} & a_{12} & a_{13}\\
0 & a_{22}^{(2)} &a_{23}^{(2)}\\
0 & a_{32}^{(2)} & a_{33}^{(2)}
\end{array}\right) \cdot
 \left(\begin{array}{l}
x_{1}\\
x_{2}\\
x_{3}
\end{array}\right)=
\left(\begin{array}{l}
b_{1}^{(2)} \\
b_{2}^{(2)} \\
b_{3}^{(2)}
\end{array}\right)
$
}
$$
$a_{22}^{(2)}=a_{22}-\displaystyle \frac{a_{21}}{a_{11}}$ . $a_{12}, a_{23}^{(2)}=\ldots, a_{32}^{(2)}=a_{32}-\displaystyle \frac{a_{31}}{a_{11}}$ . $a_{12}, a_{33}^{(2)}=\ldots,$

\begin{flushright}
$
b_{2}^{(2)}=b_{2}-\frac{a_{21}}{a_{11}}\ b_{1},\ b_{3}^{(2)}=b_{3}-\frac{a_{31}}{a_{11}}\ b_{1},
$
\end{flushright}

\end{frame}
\begin{frame}{Gaussian elimination}
W wyniku następnego etapu otrzymamy:
$$
U = \left(\begin{array}{lll}
a_{11} & a_{12} & a_{13}\\
0 & a_{22}^{(2)} &a_{23}^{(2)}\\
0 & 0 & a_{33}^{(3)}
\end{array}\right)
,\ c=\left(\begin{array}{l}
b_{1}\\
b_{2}^{(2)}\\
b_{3}^{(3)}
\end{array}\right)
$$
Ogólnie po $k$ etapach otrzymujemy:
\begin{flushright}
$
A^{(k+1)}\cdot x=b^{(k+1)}\cdot x
$

$a_{ij}^{(k+1)}=a_{ij}^{(k)}-\displaystyle \frac{a_{ik}^{(k)}}{a_{kk}^{(k)}}\cdot a_{kj}^{(k)}, i, j>k$ \fbox{1a}
$$
a_{kk}
$$
$b_{i}^{(k+1)}=b_{i}^{(k)}-\displaystyle \frac{a_{ik}^{(k)}}{a_{kk}^{(k)}}\cdot b_{k}^{(k)}, i>k$ \fbox{1b}
\end{flushright}
przy założeniu, że $a_{kk}^{(k)}\neq 0$

\end{frame}
\begin{frame}{Gaussian elimination}
W przeciwnym przypadku $\rightarrow$ row interchanges\newline W efekcie $A^{(n)}=U$ -upper triangular multiplier $l_{ik}=\displaystyle \frac{a_{ik}^{(k)}}{a_{kk}^{(k)}}, a_{kk}^{(k)}\neq 0$
\end{frame}
\begin{frame}{Relationship with LU factorization}
$$
L^{(k)}=\begin{bmatrix}
1 \\
 & \ddots & & 0 \\
 & & 1\\
 & &   -l_{k^{1}+1,k}  & \ddots\\
 0 & &  \vdots \\
 & & -l_{n,k} &  0  & 1

\end{bmatrix}
$$
- unit (1'' on diag.) lower triangular

- elementary (sometimes) lower triangular

- podobna do macierzy $I$

\fbox{1a} możemy zapisać w postaci macierzowej
$$
A^{(k+1)}=L^{(k)}\cdot A^{(k)}
$$
\begin{flushright}
{\it Zadanie}: Sprawdzić
\end{flushright}


\end{frame}
\begin{frame}{Relationship with LU factorization}
i w konsekwencji
$$
U=A^{(n)}=L^{(n-1)}\cdot L^{(n-2)}\cdot\ldots\cdot L^{(1)}\cdot A \ \ \ (*)
$$
\begin{flushright}
{\it Zadanie}: Sprawdź
\end{flushright}
łatwo pokazać, że:
$$
(L^{(k)})^{-1}=\begin{bmatrix}
1 \\
 & \ddots & & 0 \\
 & & 1\\
 & &   -l_{k^{1}+1,k}  & \ddots\\
 0 & &  \vdots \\
 & & l_{n,k} &  0  & 1

\end{bmatrix}
$$
bo $L^{(k)}\cdot(L^{(k)})^{-1}=I$
\end{frame}
\begin{frame}{Relationship with LU factorization}
Mnożąc kolejno przez $(L^{(n-1)})^{-1}, (L^{(n-2)})^{-1}$, . . . , $(L^{(1)})^{-1}$ otrzymamy:
$$
A=\underbrace{(L^{(1)})^{-1}\cdot(L^{(2)})^{-1}\cdot\ldots\cdot(L^{(n-1)})^{-1}}\cdot U\rightarrow A=L\cdot U
$$
\hspace{17mm}
=L=
$
\begin{bmatrix}
1 & & & \\
l_{21}  & 1 & & 0 \\
l_{31} & l_{32} & \ddots & 1 \\
\vdots & \vdots & l_{k+1,k} & \vdots \\
l_{n1} & l_{n2} & l_{nk} & 1
\end{bmatrix}
$
\end{frame}
\begin{frame}{Relationship with LU factorization}
$\bullet$ do przechowywania $A^{(1)}, A^{(2)}$, . . . wystarczy jedna macierz \newline
$\bullet l_{ij}$ -przechowywanie w miejscu powstająych kolumn z zerami \newline
$\bullet$ nie musimy pamiętać "1" na diagonali

$\bullet L/U$ array - w spakowanej postaci
\end{frame}