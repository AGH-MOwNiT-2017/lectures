\section{Metoda elementu skończonego}

<obrazek>

\begin{itemize}
	\item odcinek $[x_0, x_N] \rightarrow$ podział na N części
	\item na każdym $[x_i, x_{i+1}]$ przybliżamy $\varphi(x)$ przez linię prostą odcinek $[x_i, x_{i+1}] \Rightarrow$   
\end{itemize}

element

$$
u^{(i)}(x) = u_i + \frac{x - x_i}{x_{i+1} - x_i}(u_{i+1} - u_i)
$$

$$
u_x^{(i)}(x) = \frac{u_{i+1} - u_i}{x_{i+1} - x_i}
$$

przybliżony funkcjonał ma postać:

$$
I[u] = \sum_{i=0}^{N-1} \int_{x_i}^{x_{i+1}} F(x, u^{(i)}, u_x^{(i)} ) dx
$$

Stałe $u_i$ należy dobrać tak, by $I[u]$ był ekstremalny, tj.

$$
\frac{\partial I[u]}{\partial u_i} = 0, i = 1, ... , N-1
$$

liniowa aproksymacja $\Rightarrow$ proste rozwiązanie

do rozwiązania: układ równań liniowych

uogólnienia: 2D, 3D

% Dead link 
% Simple Finite Element - Applet (Jason White)
% http://mech.utah.edu/~jwhite/assign8/SimpleFE.html