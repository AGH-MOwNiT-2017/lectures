\section{Podejście wariacyjne - metoda Rayleigha-Ritza}

Przypomnienie:
rachunek wariacyjny $\rightarrow$ metody wyznaczania funkcji $y(x)$, dla której dany funkcjonał przyjmuje wartość ekstremalną

$F(x,y,y')$ - funkcja klasy $C^2[a,b]$

$\Omega_0$ - zbiór funkcji $y = y(x)$, klasy $C^1[a,b]$:

$y(x_1) = y_1$ ; $y(x_2) = y_2$ ; $y_1, y_2$ - dane liczby

funkcjonał:
$I[y] = \int_a^b F(x,y,y') dx$

<obrazek piłki do rugby>

$\rightarrow$ spośród wszystkich $y_i(x)$ należy wybrać tę, która ekstremalizuje $I[y]$

wartość $I[y]$ zależy od funkcji, a nie od liczby $\rightarrow$ stąd nazwa: \textbf{funkcjonał}

warunkiem koniecznym istnienia ekstremum $I[y]$ jest spełnienie przez $F(x,y,y')$ równania Eulera-Lagrange'a:

$$
F_y - \frac{d}{dx} F_{y'} = 0
$$

$F_y , F_{y'}$ - odpowiednie pochodne cząstkowe

łatwo sprawdzić, że:

$$
\varphi_{xx}(x) + \varphi(x) = \alpha
$$

jest równaniem Eulera-Lagrange'a funkcjonału:

$$
I[\varphi] = \int_0^{\frac{\pi}{2}} [\frac{1}{2}(\varphi_x)^2 - \frac{1}{2}\varphi^2 + \alpha \varphi] dx
$$

zamiast rozwiązywać równanie różniczkowe, szukamy funkcji:

$$
\begin{cases}
	\varphi(x) \in C^2[0, \frac{\pi}{2}] \\
	\varphi(0) = \varphi(\frac{\pi}{2}) = 1 \\
	\text{aproksymującej min } I[\varphi]
\end{cases}
$$

Niech: $\varphi_0(x), \varphi_1(x), ... , \varphi_i(x), ... \rightarrow$ dany:
\begin{itemize}
	\item ciąg liniowo niezlaeżnych funkcji
	\item tworzących na odcinku $[0, \frac{\pi}{2}]$ ukłąd zupełny
\end{itemize}

ciąg $\{\varphi_i\}$ nosi nazwę układu funkcji współrzędnych lub układu współrzędnych

$$
U_n(x) = \sum_{i=0}^{N+1} a_i \varphi_i \leftarrow \text{ aproksymacja min}
$$

$a_0, a_{N+1} \leftarrow$ uzyskujemy z warunków logicznych

$a_i, i = 1,2, ... , N$ - z warunku minimum funkcjonału $I[U_N]$

$$
\frac{\partial I[U_N]}{\partial a_j} = 0, j = 1,2, ... , N
$$

jawna postać - po podstawieniu $U_N(x)$ do $I$

w rezultacie - do rozwiązania układu N równań liniowych o N niewiadomych $a_i$

Uwagi:

\begin{itemize}
	\item takie $\varphi_i(x)$ by szybko uzyskać zbieżność aproksymacji $U_N0$ (tj. małe N)
	\item proste całkowania w I
\end{itemize}

$\rightarrow$ sprzeczne wymagania
$\rightarrow$ niebezpieczeństwo, że macierz współczynników będzie źle uwarunkowana - np. 

$$
\varphi_i = a_i \cdot x^i
$$

Układ współrzędnych:

\begin{itemize}
	\item wielomiany ortogonalne
	\item funkcje trygonometryczne
\end{itemize}

pod całką nie ma $\varphi_{xx} \rightarrow$ dla istnienia $I[\varphi]$ wystarczy, by $\varphi_x$ była odcinkami ciągła

W jakiej klasie funkcji szukać przybliżenia?

Kiedy istnieje minimum funkcjonału?

Czy zawsze dla RR istnieje odpowiedni funkcjonał?

Trudności z uwzględnieniem złożonych warunków brzegowych lub początkowych