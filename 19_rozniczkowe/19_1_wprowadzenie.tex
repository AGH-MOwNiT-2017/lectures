\section{Wprowadzenie}
	
Większość zagadnień nauki i techniki jest formułowanych w postaci równań (lub układów równań) różniczkowych z zadanymi warunkami granicznymi (początkowymi lub brzegowymi)

większość przypadków - brak rozwiązań analitycznych
stąd - konieczność przybliżonego znajdowania rozwiązań

Dalej - metody algebraizacji równań różniczkowych na przykłądzie problemu brzegowego dla równania różniczkowgo zwyczajnego:

$$
\begin{cases}
	\varphi_{xx}(x) + \varphi(x) = \alpha \\
	\varphi(0) = \varphi(\frac{\pi}{2}) = \alpha \\
	x \in [0,\frac{\pi}{2}]
\end{cases}
$$

Dokładnie jedno rozwiązanie:
$$
\varphi(x) = (1- \alpha) \cdot (sin(x) + cos(x) + \alpha)
$$