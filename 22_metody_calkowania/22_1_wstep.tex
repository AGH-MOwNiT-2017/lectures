\section{Wstęp}
%%%%%%%%%%%%%%%%%%%%%%%%%%
\begin{frame}{Model}
  \begin{center}
    \fbox{$\left\{\begin{array}{ll}
    \frac{du}{dt} + f(u,t) = 0 &\textrm{u = u(t)}\\
    \textrm{warunki początkowe:} & \textrm{$u(t^0) = u^0$}
    \end{array}\right.$}
  \end{center}

  - $\frac{df}{du} > 0$ - równanie typu ``rozpadu" \newline
  - $\frac{df}{du} < 0$ - równanie typu ``wzrostu \newline
  - $u$ zespolone - równanie typu ``oscylacyjnego" \par
  \qquad\qquad gdy - $u$,$f$ zespolone $\Rightarrow$ \underline{para równań} \newline
  \begin{itemize}
    \item schematy całkowania po czasie $\Rightarrow$ przejście do równań różnicowych
    \item właściwości schematów
  \end{itemize}
  Przenosi się to na:
  \begin{itemize}
    \item układy równań różniczkowych zwyczajnych
    \item układy równań różniczkowych cząstkowych
  \end{itemize}
\end{frame}
%%%%%%%%%%%%%%%%%%%%%%%%%%
\begin{frame}{Model c.d.}
  \textbf{Równanie modelowe można scałkować na siatce czasowej:}
  $$\Delta t = t^{n+1}+t^n$$
  $$u^{n+1} = u^n - \underbrace{\int_{t^n}^{t^{n+1}}f(u,t)dt}$$
  \begin{flushright}
    różne aproksymacje $\Rightarrow$ metody całkowania
  \end{flushright}
\end{frame}
%%%%%%%%%%%%%%%%%%%%%%%%%%