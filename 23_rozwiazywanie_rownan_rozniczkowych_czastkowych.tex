\PassOptionsToPackage{lowtilde}{url}
\documentclass[aspectratio=43,english]{beamer} %If you want to create Polish presentation, replace 'english' with 'polish' and uncomment 3-th line, i.e., '\usepackage{polski}'
\usepackage[utf8]{inputenc}
\usepackage{polski} %Uncomment for Polish language
\usepackage{babel}
\usepackage{listings} %We want to put listings

\mode<beamer>{ 	%in 'beamer' mode
	\hypersetup{pdfpagemode=FullScreen}		%Enable Full screen mode
	\usetheme{JuanLesPins} 		%Show part title in right footer
	%\usetheme[dark]{AGH}                 		%Use dark background
	%\usetheme[dark,parttitle=leftfooter]{AGH}  	%Use dark background and show part title in left footer
}
\mode<handout>{	%in 'handout' mode
	\hypersetup{pdfpagemode=None}
	\usepackage{pgfpages}
  	\pgfpagesuselayout{4 on 1}[a4paper,border shrink=5mm,landscape]	%show 4 slides on 1 page
  	\usetheme{boxes}
  	\addheadbox{structure}{\quad\insertpart\hfill\insertsection\hfill\insertsubsection\qquad} 	%content of header
 	\addfootbox{structure}{\quad\insertauthor\hfill\insertframenumber\hfill\insertsubtitle\qquad} 	%content of footer
}

\AtBeginPart{ %At begin part: display its name
	\frame{\partpage}
}


%%%%%%%%%%% Configuration of the listings package %%%%%%%%%%%%%%%%%%%%%%%%%%
% Source: https://en.wikibooks.org/wiki/LaTeX/Source_Code_Listings#Using_the_listings_package
%%%%%%%%%%%%%%%%%%%%%%%%%%%%%%%%%%%%%%%%%%%%%%%%%%%%%%%%%%%%%%%%%%%%%%%%%%%%
\lstset{ %
  backgroundcolor=\color{white},   % choose the background color
  basicstyle=\footnotesize,        % the size of the fonts that are used for the code
  breakatwhitespace=false,         % sets if automatic breaks should only happen at whitespace
  breaklines=true,                 % sets automatic line breaking
  captionpos=b,                    % sets the caption-position to bottom
  commentstyle=\color{green},      % comment style
  deletekeywords={...},            % if you want to delete keywords from the given language
  escapeinside={\%*}{*)},          % if you want to add LaTeX within your code
  extendedchars=true,              % lets you use non-ASCII characters; for 8-bits encodings only, does not work with UTF-8
  frame=single,	                   % adds a frame around the code
  keepspaces=true,                 % keeps spaces in text, useful for keeping indentation of code (possibly needs columns=flexible)
  keywordstyle=\color{blue},       % keyword style
  morekeywords={*,...},            % if you want to add more keywords to the set
  numbers=left,                    % where to put the line-numbers; possible values are (none, left, right)
  numbersep=5pt,                   % how far the line-numbers are from the code
  numberstyle=\tiny\color{gray},   % the style that is used for the line-numbers
  rulecolor=\color{black},         % if not set, the frame-color may be changed on line-breaks within not-black text (e.g. comments (green here))
  showspaces=false,                % show spaces everywhere adding particular underscores; it overrides 'showstringspaces'
  showstringspaces=false,          % underline spaces within strings only
  showtabs=false,                  % show tabs within strings adding particular underscores
  stepnumber=2,                    % the step between two line-numbers. If it's 1, each line will be numbered
  stringstyle=\color{cyan},        % string literal style
  tabsize=2,	                   % sets default tabsize to 2 spaces
  title=\lstname,                  % show the filename of files included with \lstinputlisting; also try caption instead of title
                                   % needed if you want to use UTF-8 Polish chars
  literate={?}{{\k{a}}}1
           {?}{{\k{A}}}1
           {?}{{\k{e}}}1
           {?}{{\k{E}}}1
           {�}{{\'o}}1
           {�}{{\'O}}1
           {?}{{\'s}}1
           {?}{{\'S}}1
           {?}{{\l{}}}1
           {?}{{\L{}}}1
           {?}{{\.z}}1
           {?}{{\.Z}}1
           {?}{{\'z}}1
           {?}{{\'Z}}1
           {?}{{\'c}}1
           {?}{{\'C}}1
           {?}{{\'n}}1
           {?}{{\'N}}1
}
%%%%%%%%%%%%%%%%%


\title{Metody Obliczeniowe w Nauce i Technice}
\author{dr Marian Bubak}
\date{}
\institute[AGH]{
	Department of Computer Science\\ul. Kawiory 21\\30-055 Krakow\\
	Poland\\
	\url{http://www.icsr.agh.edu.pl/~mownit/}
}


% Powtarzam oznaczenie równania "(*)" w różnych punktach jako nowe oznaczenie. Czy tak jest dobrze, czy zamienić na unikalne w obrębie prezentacji?

\subtitle{23. Rozwiązywanie równań różniczkowych cząstkowych}

\begin{document}
  \maketitle
	\begin{frame}{Plan wykładu}
		\tableofcontents
	\end{frame}

  \section{Klasyfikacja PDE}

\begin{frame}{Klasyfikacja PDE}
  \begin{block}{Oznaczenie}
    PDE -- partial differential equation (równanie różniczkowe cząstkowe)
  \end{block}
\end{frame}

\begin{frame}
  \begin{block}{Definicja -- równanie różniczkowe cząstkowe drugiego rzędu}
    Niech:\\
    $R$ -- zbiór na płaszczyźnie

    Równanie:

    \begin{multline*}
      a(x, y, u, \frac{{\partial}u}{{\partial}x}, \frac{{\partial}u}{{\partial}y}) \cdot \frac{{\partial}^2u}{{\partial}x^2} +
      2 \cdot b(x, y, u, \frac{{\partial}u}{{\partial}x}, \frac{{\partial}u}{{\partial}y}) \cdot \frac{{\partial}^2u}{{\partial}x{\partial}y} + \\
      + c(x, y, u, \frac{{\partial}u}{{\partial}x}, \frac{{\partial}u}{{\partial}y}) \cdot \frac{{\partial}^2u}{{\partial}y^2} +
      f(x, y, u, \frac{{\partial}u}{{\partial}x}, \frac{{\partial}u}{{\partial}y}) = 0 \qquad (*)
    \end{multline*}

    z warunkiem $a^2 + b^2 + c^2 \not = 0 \quad \forall_{(x,y) \in R}$

    nosi nazwę \textbf{równania różniczkowego cząstkowego drugiego rzędu}.
  \end{block}
\end{frame}

\begin{frame}
  W szczególnym przypadku:

  \begin{block}{Definicja -- równanie liniowe i słabo nieliniowe}
    Niech:
    $$a = a(x,y), \quad b = b(x,y), \quad c = c(x,y)$$

    równanie to jest \textit{liniowym}, gdy:
    $$f \equiv d(x,y) \frac{{\partial}u}{{\partial}x} + e(x,y) \frac{{\partial}u}{{\partial}y} + g(x,y)u + h(x,y)$$

    zaś \textit{słabo nieliniowym}, gdy:
    $$f \equiv f(x,y,u)$$
  \end{block}
\end{frame}

\begin{frame}
  W dowolnym $(x,y) \in R$ równanie $(*)$ jest:
  \begin{itemize}
    \item eliptyczne, gdy $b^2 - ac < 0$
    \item paraboliczne, gdy $b^2 - ac = 0$
    \item hiperboliczne, gdy $b^2 - ac > 0$
  \end{itemize}
\end{frame}

\begin{frame}
  \begin{exampleblock}{Przykłady}
      \begin{itemize}
        \item r. potencjału; Laplace'a $\rightarrow$ eliptyczne (wszędzie)
        $$\frac{{\partial}^2u}{{\partial}x^2} + \frac{{\partial}^2u}{{\partial}y^2} = 0$$

        \item r. transportu ciepła $\rightarrow$ paraboliczne
        $$\frac{{\partial}^2u}{{\partial}x^2} - \frac{{\partial}u}{{\partial}y} = 0$$

        \item r. falowe $\rightarrow$ hiperboliczne
        $$\frac{{\partial}^2u}{{\partial}x^2} - \frac{{\partial}^2u}{{\partial}y^2} = 0$$
      \end{itemize}
  \end{exampleblock}
\end{frame}


  \section{Równania eliptyczne}

\begin{frame}{Równania eliptyczne}
  \textbf{Równania eliptyczne} opisują zagadnienia równowagi zastosowania PDE

  \begin{block}{Równanie prototypowe -- równanie Laplace'a}
    $$\left\{ \begin{array}{l}
    \frac{{\partial}^2u}{{\partial}x^2} + \frac{{\partial}^2u}{{\partial}y^2} = 0 \\
    u_{xx} + u_{yy} = 0 \\
    \left. \begin{array}{l}
    \Delta u = 0 \\
    {\nabla}^2 u = 0
    \end{array} \right\} \text{nie jest sprecyzowany układ współrzędnych}
    \end{array} \right.$$
  \end{block}
  $\rightarrow$ teoria potencjału, grawitacji.
\end{frame}

\begin{frame}
  rozwiązania Laplace'a $\rightarrow$ funkcje harmoniczne

  Ważna własność funkcji harmonicznych:
  \begin{block}{Własność min-max}
    Jeżeli
    \begin{itemize}
      \item $R$ -- obszar jednospójny (simply connected)
      \item $S$ -- brzeg obszaru
      \item $U$ -- f. harmoniczna na $R$ i ciągła na $R \bigcup S$ % Czy to na pewno ma być wielkie U???
    \end{itemize}

    to \textit{$u$ przyjmuje największą i najmniejszą wartość na brzegu $S$}.
  \end{block}
\end{frame}


  \subsection{Warunki graniczne dla r. Laplace'a}

\begin{frame}{Warunki graniczne dla r. Laplace'a}
  % Sztuczny slajd, jest tu tylko po to, by zaprezentować najpierw tytuł podsekcji, a następnie podpodsekcji – inaczej nikt się nie zorientuje, że zagadnienie Neumanna to też warunek graniczny Laplace'a.
  Rozważymy:
  \begin{itemize}
    \item zagadnienie Dirichleta
    \item zagadnienie Neumanna
    \item zagadnienie Robina
  \end{itemize}
\end{frame}

\subsection*{Warunki graniczne dla r. Laplace'a -- zagadnienie Dirichleta}
\subsubsection{Zagadnienie Dirichleta}

\begin{frame}{Zagadnienie Dirichleta}
  \begin{block}{Zagadnienie Dirichleta}
    mając:
    \begin{itemize}
      \item $G$ -- ograniczony zbiór punktów
      \item $R$ -- wnętrze $G$; $R$ -- jednospójny
      \item $S$ -- brzeg obszaru $R$; $S$ -- odcinkami regularny
      \item $f(x,y)$ -- dana funkcja ciągła na $S$
    \end{itemize}
    należy znaleźć funkcję $u(x,y)$:
    \begin{itemize}
      \item określoną i ciągłą na $R \bigcup S$
      \item identyczną z $f(x,y)$ na $S$
      \item harmoniczną na $R$
    \end{itemize}
  \end{block}
\end{frame}

\begin{frame}
  \begin{columns}
    \column{0.45 \linewidth}
      \includegraphics[width = \linewidth]{img/23/dirichlet}
    \column{0.45 \linewidth}
      \begin{itemize}
        \item graf $f(x,y)$ -- w 3D -- zamknięta krzywa
        \item graf $u(x,y)$ -- powierzchnia nad $R \bigcup S$, zawiera $f$
        \item $f$ -- brzeg
      \end{itemize}
  \end{columns}
\end{frame}

\begin{frame}
  Dowodzi się:
  \begin{itemize}
    \item istnienia i jednoznaczności rozwiązania zagadnienia Dirichleta
    \item $S$: prostokąt -- rozwiązanie -- szeregi Fouriera, \\
    okrąg, elipsa -- całka Poissona, szeregi Fouriera \\
    (i gdy można zastosować przekszt. konforemne) \\
    ale: \textit{często rozwiązania analityczne -- wolnozbieżne}
    % Powyższe zdecydowanie powinno być zagnieżdżoną listą wypunktowaną, ale nie rozumiem struktury logicznej – czego dotyczy nawias?
  \end{itemize}

  W większości przypadków -- brak rozwiązań analitycznych (zamkniętych).
\end{frame}

\subsection*{Warunki graniczne dla r. Laplace'a -- zagadnienie Neumanna}
\subsubsection{Zagadnienie Neumanna}

\begin{frame}{Zagadnienie Neumanna}
  \begin{block}{Zagadnienie Neumanna}
    Wyznaczyć $u(x,y)$:
    \begin{itemize}
      \item ciągłą na $R \bigcup S$
      \item harmoniczną na $R$
      \item taką, że jej pochodna w kierunku normalnej wewnętrznej w~każdym punkcie $P$ brzegu $S$ przyjmuje zadane wartości:
      $$\frac{{\partial}u(x,y)}{{\partial}n} = g(x,y)$$
    \end{itemize}
  \end{block}
  Zagadnienie ma jednoznaczne rozwiązanie z dokładnością do stałego składnika.
\end{frame}

\subsection*{Warunki graniczne dla r. Laplace'a -- zagadnienie mieszane; Robina; trzeciego rodzaju}
\subsubsection{Zagadnienie mieszane; Robina; trzeciego rodzaju}

\begin{frame}{Zagadnienie mieszane; Robina; trzeciego rodzaju}
  \begin{block}{Zagadnienie mieszane; Robina; trzeciego rodzaju}
    $$\frac{{\partial}u}{{\partial}n} + H \cdot (u - h) = 0$$
    gdzie $H$,$h$ -- zadane funkcje
  \end{block}
  Powyższe zag. \textit{zag. brzegowe wewnętrzne} % Co za zag. zag.??? Mają być dwa??? (zagwarantuje, zagadnienie)??? A może należy zatytułować ten blok "zagadnienia brzegowe wewnętrzne" zamiast "zag. mieszane (...)"???
\end{frame}

\begin{frame}
  \begin{block}{Zagadnienia brzegowe zewnętrzne}
    Poszukiwana funkcja $u$ powinna być harmoniczną w obszarze nieograniczonym, położonym na zewnątrz powierzchni $S$.
  \end{block}
\end{frame}

\begin{frame}
  Podejście -- transformacja inwersja
  \begin{columns}
    \column{0.45 \linewidth}
      \includegraphics[width = \linewidth]{img/23/inwersja}
    \column{0.45 \linewidth}
      $$|0P| \cdot |0Q| = 1$$
  \end{columns}

  \vspace{5px}

  Zagadnienia brzegowe zewnętrzne mają jednoznaczne rozwiązania; brak ogólnej metody wyznaczania rozwiązań analitycznych.
\end{frame}


  \subsection{Przybliżenie równania Laplace'a układem równań różnicowych}

% Pierwszy link odnosi się do pliku na komputerze autora oryginalnego PDFa lub do lokalizacji na serwerze Bubaka, do której nie mamy dostępu (w co wątpię).
% Drugi link jest martwy, odpowiedni wydział odpowiedniej uczelni ma już nową stronę internetową, na której nie znalazłem podanych treści.

\begin{frame}{\large{Przybliżenie równania Laplace'a układem równań różnicowych}} % Ręcznie zmniejszona czcionka, do dogadania, czy lepiej tak, czy na dwie linie.
  \centerline{\includegraphics[width = 0.5 \linewidth]{img/23/przyblizanie}}
  $h>0, \qquad 0<h_i \le h, \qquad i = 1,2,3,4$ \\
  $u_i$  -- $u$ w punkcie $i$
\end{frame}

\begin{frame}
  Wyznaczamy parametry $\alpha_i$, $i=0, \dots ,4$ takie, by w $(x,y)$:
  $$u_{xx} + u_{yy} \equiv \alpha_0 \cdot u_0 + \alpha_1 \cdot u_1 + \alpha_2 \cdot u_2 + \alpha_3 \cdot u_3 + \alpha_4 \cdot u_4 \qquad (*)$$

  Do $(*)$ podstawiamy rozwinięcia $u$ w szereg Taylora wokół $(x,y)$:
  $$ \begin{array}{l}
    u_1 = u_0 + u_x \cdot h_1 + \frac{1}{2} u_{x x} h_1^2 + O(h_1^3) \\
    u_2 = \ldots
  \end{array} $$

  Wtedy:
  \begin{multline*}
    u_{x x} + u_{y y} \equiv u_0 \cdot (\alpha_0 + \alpha_1 + \alpha_2 + \alpha_3 + \alpha_4) + u_x \cdot (h_1 \alpha_1 - h_3 \alpha_3) + \\
    + u_y \cdot (\alpha_2 h_2 - \alpha_4 h_4) + \frac{1}{2} u_{x x} \cdot (h_1^2 \alpha_1 + h_3^2 \alpha_3) + \\
    + \frac{1}{2} u_{y y} \cdot (h_2^2 \alpha_2 + h_4^2 \alpha_4) + \sum_{i=1}^4 O(h_i^3) \qquad (**)
  \end{multline*}
\end{frame}

\begin{frame}
  Porównując ze sobą odpowiednie współczynniki, a następnie rozwiązując układ pięciu równań, mamy:
  \begin{block}{}
    $$ \begin{array}{l}
    \alpha_0 = -2 \cdot \left[ \frac{1}{h_1 h_3} + \frac{1}{h_2 h_4} \right], \\
    \alpha_1 = \frac{2}{h_1 (h_1 + h_3)}, \\
    \alpha_2 = \frac{2}{h_2 (h_2 + h_3)}, \\ % <- Indeksy wyglądają nieco podejrzanie, ale nie przeliczałem.
    \alpha_3 = \frac{2}{h_3 (h_1 + h_3)}, \\
    \alpha_4 = \frac{2}{h_4 (h_2 + h_4)}
    \end{array}$$
  \end{block}

  Zastąpienie równania Laplace'a przybliżeniem $(*)$ opiera się na:
  $$ \lim_{h \rightarrow 0} \sum_{i=1}^4 [O(h_i^3)] = 0$$ % W zasadzie w oryginale autor uparcie stosuje 0() zamiast O()...
\end{frame}

\begin{frame}
  Dla szczególnego, ważnego, przypadku:
  $$ h_1 = h_2 = h_3 = h_4 = h $$

  mamy:
  $$ -4 u_0 + u_1 + u_2 + u_3 + u_4 = 0 \qquad (***) $$

  z $(***)$ widać własność min-max:
  $$ u_0 = \frac{1}{4} (u_1 + u_2 + u_3 + u_4) $$

  czyli:
  \begin{block}{}
    $$ \min[u_1, u_2, u_3, u_4] \le u_0 \le \max[u_1, u_2, u_3, u_4] $$
  \end{block}

  Przedstawiony sposób transformacji PDE na równania różnicowe -- \textbf{uniwersalny}.
\end{frame}


  \subsection{Konstruowanie siatki}

\begin{frame}{Konstruowanie siatki}
  Dyskretyzacja zbioru punktów $R \bigcup S$
  $$ \left\{ \begin{array}{ll}
  (\overline{x}, \overline{y}) & \text{-- dowolny, ustalony punkt płaszczyzny} \\
  h > 0 & \text{-- rozmiar siatki (grid size)}
  \end{array} \right.$$
  \centerline{\includegraphics[width = \textwidth]{img/23/siatka}}
\end{frame}

\begin{frame}
  $$ \left\{ \begin{array}{ll}
  \text{zbiór punktów:} & \begin{array}{l} (\overline{x} + p \cdot h, \overline{y} + q \cdot h), \quad p,q = 0, \pm 1, \pm 2, \dots \\
    \rightarrow \text{zbiór punktów siatki płaskiej} \end{array}\\
  \text{zbiór linii:} & \left. \begin{array}{l}
    x = \overline{x} + p h \rightarrow \text{wertykalnych} \\
    y = \overline{y} + q h \rightarrow \text{horyzontalnych}
  \end{array} \right\} \begin{array}{l} \text{karta płaska} \\ \text{(planar lattice)} \end{array}
  \end{array} \right. $$
  $\rightarrow$ pokrywają całą płaszczyznę.
\end{frame}

\begin{frame}
  \begin{block}{}
    \begin{itemize}
      \item $R_h$ -- siatka wewnętrzna (interior grid): te punkty siatki, które należą do $R$
      \item $S_h^*$ -- punkty wspólne karty płaskiej i brzegu $S$
      \item $G_h^* = R_h \bigcup S_h^*$
      \item 4 sąsiedzi $(x,y) \in R_h \rightarrow$ 4 punkty $\in G_h^*$ najbliższe $(x,y)$  \\
w 4 kierunkach
      \item $G_h$ -- podzbiór $G_h^*$ zawierający każdy punkt $R_h$ i 4 jego sąsiadów
      \item Brzeg siatki $S_h = G_h - R_h$
    \end{itemize}
  \end{block}
\end{frame}


  \subsection{Przykład}

\begin{frame}{Przykład}
  \centerline{\includegraphics[height = 0.85 \textheight]{img/23/przyklad}}
\end{frame}

\begin{frame}
  \begin{block}{Oznaczenia}
    \begin{itemize}
      \item Czworokąt:
            $$(0,0), (7,0), (2,5), (0,4)$$
      \item $R$ -- wnętrze czworokąta
      \item $S$ -- brzeg czworokąta
      \item $S_i$, gdzie $i=1,\dots , 4$ -- boki czworokąta $\rightarrow$ brzeg $S$
    \end{itemize}
  \end{block}
\end{frame}

\begin{frame}
  Niech:
  \begin{itemize}
    \item $(\overline{x},\overline{y}) = (0,0);$
    \item $h=2$
  \end{itemize}

  Wówczas:
  \begin{itemize}
    \item Siatka wewnętrzna \begin{equation}R_h = \{(2,2),(2,4),(4,2)\} \end{equation}
    \item Punkty wspólne karty płaskiej i brzegu $S$: $S_h^* = S_i \bigcup S_2 \bigcup \text{(4 $\bigcirc$ i 1 $\square$ na $S_3$)}$
    \item Brzeg siatki:$$S_h = \{ (2,0),(4,0),(0,2),(5,2),(4,3),(3,4),(0,4),(2,5) \}\quad (\bigcirc)$$
  \end{itemize}
\end{frame}

\begin{frame}
  \begin{block}{Algorytm rozwiązywania zagadnienia Dirichleta}
    \begin{description}
      \item[krok 1]
        Dla ustalonych $h>0, (\overline{x},\overline{y})$
        \begin{itemize}
          \item tworzymy $R_h$ o $m$ punktach,
          \item tworzymy $S_h$ o $n$ punktach,
          \item numerujemy $R_h$ liczbami całkowitymi $[0,m]$ narastająco od lewej do prawej, z dołu do góry,
          \item numerujemy $S_h$ liczbami całkowitymi $[m+1, n+1]$ dowolnie.
        \end{itemize}
      \item[krok 2]
        W każdym $P_k(x,y) \in S_h$ podstawiamy $u_k = f(x,y)$.
    \end{description}
  \end{block}
\end{frame}

\begin{frame}
  \begin{block}{Algorytm rozwiązywania zagadnienia Dirichleta -- c.d.}
    \begin{description}
      \item[krok 3]
        W każdym $(x,y) \in R_h$ zapisujemy różnicowy odpowiednik równania Laplace'a:
        \begin{multline*}
          $$-2 \cdot ( \frac{1}{h_1 h_3} + \frac{1}{h_2 h_4}) \cdot u(x,y) + \\
          + \frac{2}{h_1 (h_1 + h_3)} u(x+h_1,y) + \frac{2}{h_2 (h_2 + h_4)} u(x,y + h_2) + \\
          + \frac{2}{h_3 (h_1 + h_3)} u(x-h_3,y)+\frac{2}{h_4 (h_2 + h_4)} u(x,y - h_4)=0$$
        \end{multline*}
        jeżeli zaś punktem sąsiednim $(x,y)$ jest punkt $\in S_{h^*}$
        to $u$ w punkcie sąsiednim zastępujemy przez wartość $f(x,y) \rightarrow$ krok 2.

        Otrzymujemy układ $m$ równań o $m$ niewiadomych.
      \item[krok 4]
        Rozwiązanie układu równań.
    \end{description}
  \end{block}
\end{frame}

\begin{frame}
  \begin{block}{Algorytm rozwiązywania zagadnienia Dirichleta -- c.d.}
    \begin{description}
      \item[krok 5]
        Dyskretna funkcja $u_i$, gdzie $i=1,2, \dots ,m+n$, określona tylko na $R_h + S_h$, reprezentuje przybliżone rozwiązanie zagadnienia Dirichleta.
    \end{description}
  \end{block}

  \begin{block}{Stosowanie powyższego algorytmu opiera się na poniższych faktach}
    \begin{enumerate}
      \item przybliżone rozwiązanie zagadnienia Dirichleta istnieje i jest jednoznaczne,
      \item dla szerokiej klasy zagadnień rozwiązanie numeryczne jest zbieżne do analitycznego z $h \rightarrow 0$,
      \item otrzymany układ równań może być rozwiązany metodą SOR dla dowolnego przybliżenia początkowego z $ \omega \in (0,2)$; dla pewnych klas zagadnień można znaleźć optymalne wartości $\omega$ (najszybsza zbieżność procesu iteracyjnego).
    \end{enumerate}
  \end{block}
\end{frame}

\begin{frame}
  \begin{exampleblock}{Przykład}
    $S$: \includegraphics[width = 0.2 \textwidth]{img/23/prostokat}
    $$ \omega = \frac{2}{1 + \sqrt{1 - \lambda ^2}}, \quad \lambda = \frac{1}{2} \left( \cos \frac{\pi \cdot h}{a} + \cos \frac{\pi \cdot h}{b} \right)$$
  \end{exampleblock}

  \begin{alertblock}{Uwaga}
    Układ równań liniowych ma macierz diagonalnie dominującą $\Rightarrow$ wynik uporządkowania, określonego w kroku 3: na diagonali -- współczynniki przy $u(x,y)$.
  \end{alertblock}
\end{frame}

\begin{frame}
  \textit{Powrót do przykładu}

  \begin{itemize}
    \item $R$ -- wnętrze czworoboku
    \item $S$ -- bok czworoboku
    \item zagadnienie Dirichleta $z$
    \item $f(x,y) = x^2 - y^2$ na $S$
    \item $(\overline{x}, \overline{y}) = (0,0), h=2$
    \item $R_h$: $1,2,3$
    \item $S_h$: $4,5, \dots , 11$
  \end{itemize}
\end{frame}

\begin{frame}
  \centerline{\includegraphics[height = 0.85 \textheight]{img/23/przyklad2}}
\end{frame}

\begin{frame}
  $$ u_4 = 4, u_5 = 16, u_6 = -4, u_7 = 21 $$
  $$ u_8 = 7, u_9 = -16, u_{10} = -7, u_{11} = -21 $$

  \begin{multline*}
    $$ -2 (\frac{1}{h_1 h_3} + \frac{1}{h_2 h_4}) u_0 + \\
    + \frac{2 u_1}{h_1 (h_1 + h_3)} + \frac{2 u_2}{h_2 (h_2 + h_4)} + \frac{2 u_3}{h_3 (h_1 + h_3)} + \frac{2 u_4}{h_4 (h_2 + h_4)} = 0 $$
  \end{multline*}
\end{frame}

\begin{frame}
    $$ \left\{ \begin{array}{ll}
    -u_1 + \frac{1}{4} u_2 + \frac{1}{4} u_3 + \frac{1}{4} (-4) + \frac{1}{4} \cdot 4 & =0 \\
    -2 u_2 + \frac{2}{3} \cdot 21 + \frac{2}{3} \cdot 7 + \frac{1}{3} u_1 + \frac{1}{3} \cdot 16 & =0 \\
    -2 u_3 + \frac{2}{3} (-7) + \frac{2}{3} (-21) + \frac{1}{3} (-16) + \frac{1}{3} u_1 & =0
    \end{array} \right. $$

  $$ \left\{ \begin{array}{llll}
  - u_1 & + \frac{1}{4} u_2 & +\frac{1}{4} u_3 & =0 \\
  \frac{1}{3} u_1 & -2 u_2 && =-24 \\
  \frac{1}{3} u_1 && -2 u_3 & =24
  \end{array} \right. $$

  Rozwiązanie: $u^T = (0, 12, -12)$.
\end{frame}


  \section{Równania paraboliczne}

\begin{frame}{Równania paraboliczne -- Wstęp}
  \begin{block}{Prototyp: równanie transportu ciepła}
    $$ u_{x x} = u_t \quad (*) $$
  \end{block}
\end{frame}


  \subsection{Initial value problem}

\begin{frame}{Initial value problem}
  \begin{description}
    \item[Dane:]
      $f(x)$ ciągła dla wszystkich $x$
    \item[Szukana:]
      $u(x,t)$
      \begin{itemize}
        \item określona i ciągła \\ dla $-\infty < x < \infty, t \ge 0$
        \item spełniająca $(*)$ \\ dla $-\infty < x < \infty, t > 0$
        \item spełniająca $u(x,0) = f(x)$ \\ dla $-\infty < x < \infty, t = 0$
      \end{itemize}
  \end{description}
\end{frame}

\begin{frame}
  \centerline{\includegraphics[height = 0.85 \textheight]{img/23/ivp}}
  (half-plane)
\end{frame}

\begin{frame}
  Rozwiązanie $\rightarrow$ całka Fouriera
  \begin{alertblock}{Problem}
    \begin{itemize}
      \item przypadek nieliniowy $\rightarrow$ ?
      \item kłopoty z wyznaczeniem wartości w $(x,t)$
    \end{itemize}
  \end{alertblock}
\end{frame}


  \subsection{Initial boundary problem}

\begin{frame}{Initial boundary problem}
  \begin{description}
    \item[Dane:]
      \begin{itemize}
        \item stała $a>0$
        \item trzy ciągłe funkcje: $g_1(t)$, $g_2(t)$, $f(x)$ dla $ \left\{ \begin{array}{l} t \ge 0 \\ 0 \le x \le a \end{array} \right. $
      \end{itemize}
    \item[Szukana:]
      $u(x,t)$:
      \begin{itemize}
        \item określona i ciągła \\ dla $0 \le x \le a, t \ge 0$,
        \item spełniająca $(\ref{7})$ \\ dla $0 < x < a, t > 0$,
        \item spełniająca:
        $$ \left\{ \begin{array}{lrl}
        u(x,0) = f(x), & 0 \le x \le a & \text{initial condition} \\
        u(0,t) = g_1(t), & t \ge 0 & \text{boundary condition} \\
        u(a,t) = g_2(t), & t \ge 0 & \text{boundary condition}
        \end{array} \right. $$
      \end{itemize}
  \end{description}
\end{frame}

\begin{frame}
  \centerline{\includegraphics[height = 0.85 \textheight]{img/23/ibp}}
\end{frame}

\begin{frame}
  Rozwiązanie $\rightarrow$ szeregi Fouriera

  \begin{alertblock}{Problem}
    Ten sam problem jak poprzednio...
  \end{alertblock}
\end{frame}


  \section{Bibliografia}

  \begin{frame}{Bibliografia}
      \begin{thebibliography}{9}
          \setbeamertemplate{bibliography item}[online]
      				\bibitem{dos}{Mathematics Archives \newblock  Windows/MSDos Software Collection for PDE \newblock \url{http://archives.math.utk.edu/software/msdos/partial.diff.equations/}}
      \end{thebibliography}
  \end{frame}

\end{document}
