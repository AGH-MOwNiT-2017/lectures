\section{Pierwsze zastosowania Simulated Annealing}

%%%%%%%%%%%%%%%%
	\begin{frame}{Pierwsze zastosowania Simulated Annealing}
		\begin{itemize}		
			\item \textbf{Kirkpatrick}\\				
				TSP - 3000 random cities (dokładne rozwiązanie $\leq$ 318) miasta w klastrach:
				\begin{itemize}
					\item duże T - optymalna droga między klastrami
					\item małe T - optymalna droga wewnątrz klastrów
				\end{itemize}	
				$\Rightarrow$ "divide and conquer" behaviour $\rightarrow$ podział zagadnienia na różne skale
			\item \textbf{Kirkpatrick, Gelatt}\\				
				Optymalizacja rozłożenia mikroukładów elektronicznych na 1 lub więcej chipach, łączenie chipów.
		\end{itemize}
	\end{frame}
	
	\begin{frame}{Pierwsze zastosowania Simulated Annealing}
		\begin{itemize}
			\item \textbf{Vecchi, Kirkpatrick}\\				
				Optimal wiring (między VLSI)
					\begin{itemize}
						\item min length
						\item min bends
						\item no crowding
					\end{itemize}
					
			\item \textbf{Kenneth Wilson, Dean Jacobs, Jan Prins - Cornell University}\\				
				Algorytm Metropolisa do optymalizacji kodu komputerowego\\ ($\approx$ iteracyjne przestawianie komend)
		\end{itemize}
	\end{frame}

	\begin{frame}{Dalsze zastosowania Simulated Annealing}
		\begin{itemize}
			\item \textbf{Biologia i chemia molekularna}\\
				Przykłady:
					\begin{itemize}
						\item Optymalizacja struktury cząsteczki białka zbudowanej metodami modelowania molekularnego
						\item Uściślanie struktur rozwiązanych metodą dyfrakcji rentgenowskiej
						\item Przejście "od więzów do struktury" przy wyznaczaniu struktury przestrzennej białka metodą NMR
						\item Proces selekcji potencjalnych leków przez dokowanie
					\end{itemize}
				Na podstawie \url{https://bioinfo.mol.uj.edu.pl/courses/AppliedModelling/LectureSchedule?action=AttachFile&do=get&target=w3.pdf}
		\end{itemize}
	\end{frame}


