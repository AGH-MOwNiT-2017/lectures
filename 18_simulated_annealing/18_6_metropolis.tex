\section{Schemat Metropolisa}
Podstawa metody Monte Carlo symulacji molekularnej

<obrazek>

<równanie> (ale te po średniku to rozdzielić)

$$
Delta \vec{r} = \delta(\vec{i} \cdot a_x + \vec{j} \cdot a_y  + \vec{k} \cdot a_z)
$$

\begin{center}
$\delta$ - wybrane z góry $\approx 10^{-11}$ m
\end{center}

$$
a_i \in (-1, 1), Unif
$$

<obrazek>

$a_i$, R - liczby pseudolosowe

$\delta\phi$ - zmiana energii potencjalnej układy w wyniku przesunięcia molekuły m

Charakterystyka schematu Metropolisa
- $T \nearrow$ - łatwiej akceptowalne kroki z $H \nearrow (E) \rightarrow$ a to możliwość opuszczenia stanu metastabilnego (lokalnego minimum).
- zmiany $H \searrow$ są akceptowalne zawsze

Po wielu krokach system $\rightarrow$ stan równowagi termodynamicznej z parametrami oscylującymi wokół wartości średnich zgodnie z rozkładem Boltzmanna

- schemat Metropolisa $\rightarrow$ łańcuch Markowa

<obrazek>

Zasada równowagi szczegółowej (detailed balance, microscopic reversibility)

$$
p_2 \cdot = p_1 \cdot e \cdot - \dfrac{E_2 - E_1}{kT}
$$

$$
\frac{p1}{p2} = \dfrac{e^{-\frac{E_1}{kT}}}{e^{-\frac{E_2}{kT}} \rightarrow p_i ~ \underbrace{czynnik prawdopodobieństwa Boltzmanna}{e^{\frac{E_i}{kT}}}
$$

$$
P(c_i) ~ e^{\frac{E_i}{kT}}
$$

$\rightarrow$ importance sampling: konfiguracje generowane nie losowo - ale z zadanym rozkładem (50-70 % akceptowanych)
Notatka nt. niejednorodnych łańcuchów Markowa w SA, dużo bibliografii

% whaaaat
% http://www.pz.zgora.pl/discuss/al15_2/a6.htm

Uwagi praktyczne:

duże $\Delta\vec{r} : \rightarrow$
- niski poziom akceptacji
- szybsze próbkowanie przestrzeni konfiguracyjnej

Dla potencjałów parowych, krótkozasięgowych:
- przesuwamy tylko 1 cząstkę
- dla określenia $\Delta\phi$ - tylko najbliżsi sąsiedzi

W przypadku cieczy:
- każda cząsteczka $\approx 10^3$ przemieszczeń
- w modelu $\approx 10^3$ cząsteczek

Force-biased displacement
- wszystkie cząsteczki przesuwane równocześnie przeciwnie do gradietu potencjału
