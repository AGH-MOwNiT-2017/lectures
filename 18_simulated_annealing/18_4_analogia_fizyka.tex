\section{Fizyka statystyczna i optymalizacja - analogie}

%%%%%%%%%%%%%%%% 
	\begin{frame}{Fizyka statystyczna i optymalizacja - analogie}
		\begin{block}{Fizyka statystyczna}
				uśrednione wartości dla zespołów o dużej ilości molekuł	
		\end{block}
		
		
		\begin{table}[]
		\centering
		\begin{tabular}{|l|l|}
		\hline
		\textbf{Mechanika statystyczna}                                                                                                 & \textbf{Optymalizacja}                                                                  \\ \hline
		wiele oddziałujących molekuł                                                                                                    & wiele parametrów                                                                        \\ \hline
		układ, zbiór położeń molekuł                                                                                                    & konfiguracja                                                                            \\ \hline
		\begin{tabular}[c]{@{}l@{}}schłodzenie do stabilnego stanu \\ niskoenergetycznego\end{tabular}                                  & \begin{tabular}[c]{@{}l@{}}znalezienie konfiguracji \\ prawie optymalnej\end{tabular}   \\ \hline
		temperatura                                                                                                                     & \begin{tabular}[c]{@{}l@{}}parametr sterujący \\ przebiegiem optymalizacji\end{tabular} \\ \hline
		hamiltonian (energia wewnętrzna)                                                                                                & funkcja celu                                                                            \\ \hline
		\begin{tabular}[c]{@{}l@{}}w hamiltonianie człony: \\ short-range attractive, \\ long-range repulsive (spin glass)\end{tabular} & \begin{tabular}[c]{@{}l@{}}współzawodniczące człony \\ w funkcji celu\end{tabular}      \\ \hline
		\end{tabular}
		\end{table}

	\end{frame}