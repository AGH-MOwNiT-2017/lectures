\section{Simulated Annealing a rozkład kanoniczny}

- własności makroskopowe z mikroskopowymi
Q - stan mikroskopowy układu
E(Q) - energia ukłądu w stanie Q
P(Q) - rozkład prawdopodobieństwa 
Rozkład mikrokanoniczny - układ izolowany

$$
P(Q) = A(Q) \cdot \delta(E - E(Q))
A(Q) = \begin{cases}
0 - \text{niemożliwe} \\
A - \text{możliwe}
\end{cases}
$$

Rokład kanoniczny - ukłąd w kontakcie cieplnym z otoczeniem
<obrazek>

$$
P(Q) = Z^{-1} \cdot e^{-\frac{E(Q)}{kT}}
$$

%$$
%\sum_Q P(Q) = Z^{-1} \cdot \sum_Q ^{-\frac{E(Q)}{kT}} = 1 \rightarrow Z = \underarrow{suma statystyczna - funkcja rozdziału $\rightarrow$ bardzo ważna!}{\sum_Q ^{-\frac{E(Q)}{kT}}}
%$$

często: $\frac{1}{kT} = \beta$

E = Var
<E> = U $\rightarrow$ energia wewnętrzna: można ją zmierzyć

$$
%<E> = \sum_Q E(Q) \cdot P(Q) = \dfrac{\sum_Q E(Q) \cdot e^{-\beta\cdot E(Q)}}{\sum_Q e^{-\beta\cdotE(Q)}} = - \frac{\partial}{\partial \beta} ln Z
$$

$$
<E> = \frac{\partial}{\partial \beta} ln Z
$$

Ciepło właściwe przy stałej objętości:

$$
C_v = \frac{d<E(T)>}{dT}
$$

ale:

$$
c_v = \frac{k}{\beta^2}[\underbrace{<E^2(T)>}_{\text{średni kwadrat energii}} - \underbrace{<E(T)>^2}]_{\text{kwadrat średniej energii}}
$$

Wzrost wartości ciepła właściwego sygnalizuje zachodzenie przemiany fazowej.
S - entropia układu:

$$
T \cdot dS = dQ - ciepło dostarczane układowi
$$

$$
V = const
$$

$$
T \cdot dS = dU = C_v \cdot dT \rightarrow \frac{ds}{dT} = \frac{C_v(T)}{T}
$$

$$
S(T) = S(T_1) - \int_T^{T_1} \frac{C_v(T')}{T'}dT'
$$

$T_1$ - temperatura, dla której entropia jest znana (np. przez aproksymację dla b. wysokich temperatur)
Poziom S $\rightarrow$ ilość minimów (stanów ustalonych)

