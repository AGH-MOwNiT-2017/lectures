\subsection{Pierwiastki rzeczywiste -- metoda Newtona-Raphsona}

\begin{frame}{Techniki wygładzania pierwiastków (root polishing) \\Pierwiastki rzeczywiste -- metoda Newtona-Raphsona}
  \begin{itemize}
    \item $p$ -- wielomian stopnia $n$,
  \end{itemize}
    Wykonujemy schemat Hornera dla punktu $t$.
    \begin{gather*}
      a_i = b_i - b_{i + 1} t \text{, } i = 0, 1, \dots, n - 1 \\
      p(x) = a_n x^n + a_{n - 1} x^{n - 1} + \ldots + a_0 =\\
      = b_0 + (x - t) q(x)\\ \\
      p'(x) = q(x) + (x - t) q'(x)\\
      p'(t) = q(t)
    \end{gather*}
    Dla $t$ -- pochodną $p'(t)$ można obliczyć przy wyliczaniu $p(t)$.
\end{frame}

\begin{frame}[fragile]
  Pojedynczy krok algorytmu:
  \begin{lstlisting}
    //Horner
    b[n] := a[n]
    c[n] := b[n]
    for k = n - 1 downto 1
      b[k] := a[k] + t * b[k + 1]
      c[k] := b[k] + t * c[k + 1]
    b[0] := a[0] + t * b[1]

    //b[0] == p(t),
    //c[1] == p'(t)
    p := b[0]
    p1 := c[1]

    if p1 == 0
      raise error("Derivative should not vanish.")

    //Newton-Raphson step
    t := t - p / p1
  \end{lstlisting}

  \textbf{Zadanie:} Sprawdzić poprawność.
\end{frame}
