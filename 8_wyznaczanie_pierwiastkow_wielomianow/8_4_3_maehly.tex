\subsection{Procedura Maehly'ego -- technika wygładzania pierwiastków}

\begin{frame}{Pro. Maehly'ego -- technika wygładzania pierwiastków}
  Zapobiega zlewaniu się w jeden dwóch różnych pierwiastków (na~etapie wygładzania) -- równocześnie unikamy deflacji.\\

  \begin{block}{Zredukowany wielomian}
    $$P_j(x) \equiv \frac{P(x)}{(x - x_1)\ldots(x - x_j)}$$
  \end{block}
    Wykorzystujemy znane pierwiastki do znalezienia kolejnych.

  $$P'_j(x) = \frac{P'(x)}{(x - x_1)\ldots(x - x_j)} - \frac{P(x)}{(x - x_1)\ldots(x - x_j)} \sum_{i=1}^j \frac{1}{x - x_i}$$
\end{frame}

\begin{frame}
  Pojedynczy krok metody Newtona-Raphsona można zapisać:
  \begin{gather*}
    x_{k+1} = x_k - \frac{P_j(x_k)}{P'_j(x_k)}\\ \\
    x_{k+1} = x_k - \frac{P(x_k)}{P'(x_k) - P(x_k) \cdot \sum_{i=1}^{j}\frac{1}{(x_k - x_i)}}
  \end{gather*}

  $\Rightarrow$ zero suppression.
\end{frame}
