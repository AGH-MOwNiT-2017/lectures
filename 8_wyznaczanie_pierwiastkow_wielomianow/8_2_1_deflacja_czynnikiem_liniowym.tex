\subsection{Deflacja czynnikiem liniowym}

\begin{frame}{Deflacja czynnikiem liniowym}
  \begin{block}{}
    \begin{equation} f(x) = (x - \lambda) \cdot \underbrace{\sum_{i=0}^{m-1} b_i x^i}_{g(x)} + f(\lambda) \label{linear} \end{equation}
  \end{block}
  Gdy $f(\lambda) = 0$, wtedy $x = \lambda$ -- pierwiastek funkcji f.

  \vspace{5px}

  Pozostałe $m - 1$ zer w $ g(x) = \sum_{i=0}^{m-1} b_i x^i $ -- deflated equation

  \vspace{5px}

  \begin{alertblock}{Uwaga}
    Następuje kumulowanie błędów: $\lambda$ z błędem, $b_i$ z błędem itd\dots
  \end{alertblock}
\end{frame}

\begin{frame}
  \textbf{Pochodna} -- Różniczkując równanie (\ref{linear})

  $$ f'(x) = (x - \lambda) \cdot g'(x) + g(x) $$
  $$ f'(\lambda) = g(\lambda) = \sum_{i=0}^{m-1} b_i \lambda^i$$
\end{frame}
