\section{Wstęp}
%%%%%%%%%%%%%%%%%%%%%%%%%
	\begin{frame}{Wstep} 
    
a) \textit {kwadratura} - przypadek specjalny zagadnienia początkowego:
$$
 I=\int_{a}^{b}f(x)dx\ \equiv I=y(b), \displaystyle \frac{dy}{dx}=f(x) , \underbrace{(a)=0}_\text{(*)}
$$
(*) -- warunek początkowy
	\end{frame}
%%%%%%%%%%%%%%%%%%%%%%%%%	    
    \begin{frame}
    
      b)
      \begin{center}
      	\includegraphics[width=0.6\linewidth]{img/6/image001.png}
      \end{center}
      
      \begin{block}{}
      kwadratury:
      
        \begin{itemize}
          \item otwarte
          \item zamknięte
          \item półotwarte
        \end{itemize}
      \end{block}
    \end{frame}
%%%%%%%%%%%%%%%%%%%%%%%%%
    \begin{frame}
    	c) \textbf{Istota}: podział [a,b] na podprzedziały $\rightarrow$ formuły złożone
        $$
	        I=\sum I_{i}
        $$
		d) \textbf{Stopień dokładności kwadratury} \newline
        \textit{E} - błąd kwadratury, {$P_{k}$} - wielomian stopnia k \newline
        stopień - liczba całkowita n $<$ 0 :
        $$
			E(P_{k})=0  \ \ \forall k\leq n,\ E(P_{n+1})\neq 0
		$$
        całkowanie, dodawanie - operacje liniowe $\Rightarrow$ zamiast $P_{k}$$\rightarrow x^{k}.$
    \end{frame}
%%%%%%%%%%%%%%%%%%%%%%%%%
	\begin{frame}{Ekstrapolacja Richardsona}
		Metoda uzyskiwania wyników o dużej dokładności przy użyciu formuł niskiego rzędu	
       
        \begin{itemize}
          \item Archimedes (200 p.n.e.)
          \item L. F. Richardson, J. A. Gaunt (1927 n.e)
        \end{itemize}
        niech	\textit{h} - krok, \textit{E} - błąd obliczeń $\rightarrow$ postać asymptotyczna
        $$
			E=\sum_{i=k}^{\infty}a_{i}\cdot h^{i},\ a_{k}\neq 0
		$$
		przy czym: $a_{i}$
         $\left\{\begin{array}{l}
  			\textnormal{mogą zależeć od \textit{f}(x),}\\
            \textnormal{\textbf{nie} zależą od h.}
        \end{array}\right\}$
   \end{frame}
   \begin{frame}
   
   obliczenia dla $h_{1}\neq h_{2}\rightarrow$ dokładna wartość $\varphi$($\varphi_{1}$, $\varphi_{2}$-wyznaczone):
		$$
        \begin{array}{l}
\varphi=\varphi_{1}(h_{1})+\sum_{i=k}^{\infty}{a_{i}}\cdot h_{1}^{i}\ \text{\textbar}\cdot h_{2}^{k}\\
\varphi=\varphi_{2}(h_{2})+\sum_{i=k}^{\infty}{a_{i}}\cdot h_{2}^{i}\ \text{\textbar}\cdot h_{1}^{k}
		\end{array}\Bigg\lvert -
        $$
        
		$$
\varphi=\frac{1}{h_{1}^{k}-h_{2}^{k}}\cdot(h_{1}^{k}\cdot\varphi_{2}-h_{2}^{k}\cdot\varphi_{1})+\sum_{i=k+1}^{\infty}a_{i}\cdot\frac{h_{1}^{k}h_{2}^{i}-h_{1}^{i}h_{2}^{k}}{h_{1}^{k}-h_{2}^{k}}
		$$
		wynik $\Rightarrow$ zwiększenie najmniejszego wykładnika potęgi \textit{h} w \textit{E} , \\
		ER pozwala poprawić wynik nawet gdy od razu wybrano metodę rzędu \textit{k} + 1.        
	\end{frame}
%%%%%%%%%%%%%%%%%%%%%%%%%
	\begin{frame}
		Szczególny przypadek (użyteczny dla kwadratur):
		$$
\varphi=N_{j-1}(h)+\sum_{j=1}^{m-1}K_{j}\cdot h^{2j}+O(h^{2m})\ ,
        $$
        co pozwala na rekurencyjne generowanie:   
        $$
N_{j}(h)=\displaystyle \frac{1}{4^{j-1}-1}\cdot[4^{j-1}\cdot N_{j-1}(\frac{h}{2})-N_{j-1}(h)],\ \ j=2, 3, . . ., m.
		$$
    \end{frame}
%%%%%%%%%%%%%%%%%%%%%%%%%




