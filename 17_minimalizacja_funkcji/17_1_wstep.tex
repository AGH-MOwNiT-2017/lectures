\section{Wstęp}

  \begin{frame}{Wstęp}
    \begin{block}{Romuald Wit}
      \begin{enumerate}
        \item Wstęp do metod optymalizacji nieliniowej - skrypt
        UJ, 1983
        \item Metody programowania nieliniowego WNT, Biblioteka
        Inżynierii Oprogramowania.
      \end{enumerate}
    \end{block}
  \end{frame}

  \begin{frame}{Wstęp}
    \begin{block}{Linki do stron związanych z minimalizacją:}
      \begin{itemize}
        \item MINUIT -- Function Minimization and Error
        Analysis\footnote{\url{http://consult.cern.ch/writeups/minuit}}
        % TODO: change numeration of footnotes from letters to numbers
        (autor:~F.~Jarnes)
        \item Metody optymalizacji\footnote{\url{http://aquarium.ia.agh.edu.pl/labs/opt/metopt.htm}}
        (autor:~Strona~prywatna)
        \item Przykłady ciekawych wielowymiarowych
        funkcji\footnote{\url{http://www.maths.adelaide.edu.au/Applied/llazausk/alife/realopt.htm}}
        (autor:~John~Smith)
      \end{itemize}
    \end{block}
  \end{frame}

  \subsection{Motywacja}
    \begin{frame}

      \begin{exampleblock}{}
        Szeroka klasa zagadnień - szukanie najmniejszej wartości
        przyjmowanej przez funkcję jednej lub wielu zmennych.\\
        Np:\\
        \begin{itemize}
          \item minimalizacja kosztów producji
          \item \ldots
          \item linia geodezyjna w ogólnej teorii względności
          \item przykład klasyczny:\\
          \textbf{estymacja}

        \end{itemize}
      \end{exampleblock}
    \end{frame}
