\section{Reprezentacja dyskretna zagadnienia początkowego}
\begin{frame}{Reprezentacja dyskretna}
  $$\frac{d \vec{u}}{dt} = L\vec{u}$$
  $1^o$\quad L $\rightarrow$ przekształcamy na operator różnicowy (siatka przestrzenna $x_j$) \newline
  $2^o$ \quad wprowadzamy siatkę czasową: $t^n$
  $$t^n = \sum_{r=2}^{n}\Delta t_r$$
  $\Delta t_r $- krok czasowy
\end{frame}
%%%%%%%%%%%%%%%%%%%%%
\begin{frame}{Całkowanie równania stanu od $t^n$ do $t^{n+1}$}
  $$u = \vec{u}$$
  $${u}^{n+1} = {u}^n + \int_{t^n}^{t^{n+1}}dt' Lu(\vec{r},t')$$
  To związek między stanami w sąsiednich chwilach $t^n$,$t^{n+1}$.
  \begin{center}
  	\qquad $u(t')$ , $t' \in [t^n,t^{n+1}]$ -- nieznane
  \end{center}
\end{frame}
\begin{frame}
  \underline{Dla małych kroków czasowych} $\Delta t = t^{n+1} - t^n$ można:
  $${u}^{n+1} = {u}^n + \int_{t^n}^{t^{n+1}}dt'\Bigg\{ \sum_{r=0}^{p-1}\bigg[\frac{dr}{d t^r}(Lu)\bigg]_{t^n}\frac{(t'-t^n)^r}{r!}+0((t'-t^n)^p)\Bigg\}$$
  $${u}^{n+1} = {u}^n + \sum_{r=1}^{p}\bigg[\frac{d^{r-1}}{d t^{r-1}}(Lu)\bigg]_{t^n}\frac{(\Delta t)^r}{r!}+0(\Delta t^{p+1})$$
  $p$ -- rząd wielkości całkowania (ze względu na krok czasowy $\Delta t$)
\end{frame}
%%%%%%%%%%%%%%%%%%%%%
\begin{frame}{Całkowanie -- praktyka}
  W praktyce -- ograniczenie do członów drugiego rzędu.
  $$u ^{n+1} = u ^n + Lu^n\Delta t+\bigg[\frac{d}{dt}(Lu)\bigg]_{t^n} \frac{(\Delta t)^2}{2}$$
  \textbf{Wyznaczanie wartości pochodnej:}
  \begin{itemize}
    \item wartości zmiennych na poprzednich poziomach czasu $t^{n-1}$
    \item wprowadzenie poziomów pośrednich
    \item użycie nieznanych $u^n+1$ (w czasie o krok późniejszym $t^{n+1}$)
  \end{itemize}
\end{frame}
\begin{frame}{Całkowanie -- praktyka cd.}
  \textbf{Szczególny przypadek:} $\rightarrow$ użycie nieznanych wartości $u^{n+1}$:
  \begin{center}
  	$u^{n+1} = u^n+Lu^n(1-\varepsilon)\Delta t+Lu^{n+1}\varepsilon \Delta t $\qquad(*)
  \end{center}
  \begin{description}
    \item - $0 \leqslant \varepsilon \leqslant 1$ - parametr interpolacji
    \item - $\varepsilon = 0$ - metoda jawna
    \item - $\varepsilon \not= 0$ - metoda niejawna
    \item - Dokładność 2-go rzędu zachowana jedynie dla $\varepsilon=\frac{1}{2}$
  \end{description}

\end{frame}
%%%%%%%%%%%%%%%%%%%%%
\begin{frame}{Całkowanie -- praktyka c.d.}
  po uporządkowaniu w (*) można zapisać:
  $${u}^{n+1} = (1-\varepsilon\cdot\Delta t\cdot L)^{-1}\cdot[1+(1-\varepsilon)\Delta t\cdot L]\cdot u^n $$
  czyli: \fbox{$\vec{u}^{n+1}=T(\Delta,\Delta t)\vec{u}^n$} \qquad por.: $\frac{du}{dt} = Lu$\newline\newline\newline
  \textbf{T - operator różnicowy}, wiąże kolejne punkty na siatce czasowej, tożsame z zagadnienie, początkowym -- ciąg rozwiązań w izolowanych punktach czasowych (T).
\end{frame}
