\section{12.5 Sens procedury iteracyjnej dla Ax=b}

\begin{frame}{Sens procedury iteracyjnej dla Ax=b}
  W każdym kroku następuje poprawianie rozwiązania.\\
  Przypadek, gdy źródłem Ax=b jest r. Poissona :
  \begin{itemize}
    \item \fbox{$\bigtriangledown^2u(x,y)=-\rho(x,y)$}
    \item $\rho(x,y)$ -- f. rozkładu źródeł
    \item $\infty$ szybkość propagacji informacji
  \end{itemize}
\end{frame}

\begin{frame}{}
  Ale r. Poissona to graniczny (stacjonarny) przypadek \emph{równania dyfuzji} :
  $$\boxed{\frac{\partial u}{\partial t} = \bigtriangledown^2u=\rho}$$
  którego jawne sformułowanie różnicowe ma postać:
  $$
  \left.
  \begin{array}{lr}
    \text{siatka t}:\\
    t\rightarrow p,\\
    t+\Delta t\rightarrow p+1
  \end{array}\right|
  \underbrace{u^{(p+1)}=u^{(p)}+\Delta t\Delta^{||}y^{(p)}}_{Mu^{(p)}}+\underbrace{\rho\Delta t}_{W=B^{-1}b}
  $$
  

  Procedura iteracyjna- jawne rozwiązanie zagadnienia opisującego zbieżność w wyimaginowanym czsie iteracji (pseudoczas)
\end{frame}

