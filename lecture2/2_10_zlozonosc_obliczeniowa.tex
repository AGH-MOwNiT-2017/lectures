\section{Złożoność obliczeniowa {\it computational complexity}}
%%%%%%%%%%%%%%%%
\begin{frame}{Złożoność obliczeniowa}
	Oprócz jakości algorytmu - ważny jego {\it koszt} - liczba działań arytmetycznych (logicznych) potrzebnych do rozwiązania zadania - algorytmy minimalizujące liczbę działań.
\end{frame}
%%%%%%%%%%%%%%%%
\begin{frame}{Rezultaty}
	\begin{enumerate}
		\item Oszacowanie złożoności obliczeniowej ,,z dołu'': jeżeli zadanie ma $n$ danych istotnych, to minimalna liczba działań arytmetycznych potrzebnych do obliczenia wyniku:
        \[
        	z(\varphi, D) \ge \frac{n}{2}
        \]
        \item Dla wielu zadań można udowodnić, że minimalna liczba działań w algorytmie numerycznie poprawnym musi być istotnie większa od liczby danych.
        \item Metody optymalne co do $z(\varphi, D)$ znane są dla niewielu, prostych zadań.
	\end{enumerate}
\end{frame}
%%%%%%%%%%%%%%%%
\begin{frame}{Złożoność obliczeniowa - przykład}
	\begin{exampleblock}{Przykład}
        \begin{itemize}
            \item prosty
            \item ilustruje konieczność myślenia {\it do końca} przy wyborze algorytmu
        \end{itemize}
        {\bf Zadanie:}
        \[
        	S = \sum_{i/1}^n (-1)^i \cdot i;
        \]
	\end{exampleblock}
\end{frame}
%%%%%%%%%%%%%%%%
\begin{frame}[fragile]{Złożoność obliczeniowa - przykład}
	\begin{exampleblock}{Przykład}
    	{\bf A1:}
    	\begin{lstlisting}[escapechar=|]
s = 0
do 1 i=1, n
    s = s+(-1)**i*i
continue \end{lstlisting}
		\vspace{-15pt}
    	{\bf A2:}
    	\begin{lstlisting}[escapechar=|]
s = 0
do 1 i=1,n,2
    s = s-1
continue
do 1 i=1,n,2
	s = s+1
continue \end{lstlisting}
		\vspace{-15pt}
	\end{exampleblock}
\end{frame}
%%%%%%%%%%%%%%%%
\begin{frame}[fragile]{Złożoność obliczeniowa - przykład}
	\begin{exampleblock}{Przykład}
    	{\bf A3:}
    	\begin{lstlisting}[escapechar=|]
s = n/2
if (mod(n,2).eq.1) s=-s \end{lstlisting}
		\vspace{-15pt}
        \begin{itemize}
        	\item ilość operacji nie zależy od $n$
            \item nie ma akumulacji błędów
            \item nie grozi overflow
        \end{itemize}
	\end{exampleblock}
\end{frame}